\chapter{开始使用}

解压缩下载好了的fkparse后,你应该会得到这些文件:

\begin{itemize}
 \item \textbf{manual.pdf}: 本手册
 \item \textbf{fkparse.exe}: 主程序
 \item \textbf{fkparser.lua}: 让生成的代码运行起来所必需的文件
 \item \textbf{example.txt}: 一个示例文本
 \item \textbf{LICENSE}: GPLv3许可证文件
\end{itemize}

除此之外,你还要确保自己拥有太阳神三国杀游戏主程序。

\section{编译运行示例文件}

fkparse是一个命令行程序,运行方式为:

\begin{verbatim}
 $ fkparse <input file>
\end{verbatim}

当然你无需打开命令行进行操作,只需要将输入文件拖动到fkparse.exe上即可。现在请将example.txt拖动到fkparse.exe上,然后你会发现文件夹中生成了example.lua。

现在让生成的代码运行起来吧,首先把自带的fkparser.lua复制到太阳神三国杀游戏目录的lua/lib文件夹下。然后再把生成的example.lua复制到太阳神三国杀游戏文件夹下的extensions文件夹下面。接下来启动QSanguosha.exe,你就可以在武将一览中找到新武将“曹孟德”了。\\

至此,你已经知道如何使用fkparse了,接下来说明example.txt这种文件所要求的语法格式。

\section{编写自己的第一个拓展文件}

首先,新建一个文本文件,就起名为study.txt吧。在之后的部分都将基于这个文件进行操作。打开study.txt,用记事本或者代码编辑器之类的东西都可。

\subsection{创建拓展包}

拓展文件首先要从创建拓展包开始。fkparse中创建拓展包的格式为:\\

\emph{拓展包 <标识符>} \\

“标识符”就是用半角单引号括起来的文本,毕竟中文不用空格进行分词。一个拓展文件中可以含有多个拓展包,下面我们来创建两个空拓展包:

\begin{verbatim}
 拓展包 '学习包1'
 拓展包 '学习包2'
\end{verbatim}

保存,然后用它去生成study.lua。将study.lua放入extensions里面,你应该能在游戏中看到这两个拓展包了。

\subsection{创建武将}

创建武将的详细的内容在下一章说明。这里我们只是简单的新建一个武将而已。

现在我们要创建一个武将,它所属于学习包1,名字是猪八戒,称号是净坛使者,神势力,24体力。那么我们现在在“拓展包 '学习包1'”下面另起一个新行,然后输入:

\begin{verbatim}
 # 神 "净坛使者" '猪八戒' 24 []
\end{verbatim}

这一段话的详细内容下一章再细说。现在保存文件,然后重新生成lua。现在你的study.txt的内容应该是像这样:

\begin{verbatim}
拓展包 '学习包1'
# 神 "净坛使者" '猪八戒' 24 []
拓展包 '学习包2'
\end{verbatim}

\section{错误处理}

因为是命令行程序,fkparse不会给你弹出一个对话框提示哪里出错了,它会把报错的信息写入生成的lua文件中。如果你的游戏因为有错误的lua文件而无法启动,可以用记事本打开lua文件,查看报错详情,然后想办法解决,或者把原文和报错的截图发到群里去问大家。
