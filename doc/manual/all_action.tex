\chapter{所有的动作语句}

先列出语句格式,再依次说明表达式类型需求,最后说返回值(如果有)。

前面有个没说过的,动作语句也可以带额外参数,毕竟一句话没法描述完全部的情况。因此对于一部分动作语句也提供了一些额外的参数,像函数那样用就行了,具体参考已有的txt。

\section{摸牌}

\begin{verbatim}
<表达式> 摸 <表达式> 张牌
\end{verbatim}

玩家类型;数字类型

\section{失去体力}

\begin{verbatim}
<表达式> 失去 <表达式> 点体力
\end{verbatim}

玩家类型;数字类型

\section{失去体力上限}

\begin{verbatim}
<表达式> 失去 <表达式> 点体力上限
\end{verbatim}

玩家类型;数字类型

\section{造成伤害}

\begin{verbatim}
<表达式> 对 <表达式> 造成 <表达式> 点伤害
\end{verbatim}

玩家类型;玩家类型;数字类型 \\

额外参数:

\begin{itemize}
  \item \verb|'伤害属性'|:数字类型,默认为\verb|'无属性'|
  \item \verb|'造成伤害的牌'|:卡牌类型,默认为\verb|nil|
  \item \verb|'造成伤害的原因'|:字符串类型,默认为空字符串
\end{itemize}

\section{受到伤害}

\begin{verbatim}
<表达式> 受到 <表达式> 点伤害
\end{verbatim}

玩家类型;数字类型 \\

额外参数同上。

\section{回复体力}

\begin{verbatim}
<表达式> 回复 <表达式> 点体力
\end{verbatim}

玩家类型;数字类型 \\

额外参数:

\begin{itemize}
  \item \verb|'回复来源'|:玩家类型,默认为\verb|nil|
  \item \verb|'回复的牌'|:卡牌类型,默认为\verb|nil|
\end{itemize}

\section{回复体力上限}

\begin{verbatim}
<表达式> 回复 <表达式> 点体力上限
\end{verbatim}

玩家类型;数字类型

\section{获得技能}

\begin{verbatim}
<表达式> 获得技能 <表达式>
\end{verbatim}

玩家类型;字符串类型

\section{失去技能}

\begin{verbatim}
<表达式> 失去技能 <表达式>
\end{verbatim}

玩家类型;字符串类型

\section{获得标记}

\begin{verbatim}
<表达式> 获得 <表达式> 枚 <字符串> [隐藏] 标记
\end{verbatim}

玩家类型;数字类型

<字符串>表示标记的名称。

注:“隐藏”可填可不填,如果写了是隐藏标记的话,获得标记将不会显示在战报中,也没有图片,可以借此实现技能发动次数控制等。

\section{失去标记}

\begin{verbatim}
<表达式> 失去 <表达式> 枚 <字符串> [隐藏] 标记
\end{verbatim}

玩家类型;数字类型

\section{统计标记数量}

\begin{verbatim}
<表达式> <字符串> [隐藏] 标记数量
\end{verbatim}

玩家类型

返回:数字类型

\section{询问选择选项}

\begin{verbatim}
<表达式> 从 <表达式> 选择一项
\end{verbatim}

玩家类型;数组(字符串类型) \\

额外参数:

\begin{itemize}
  \item \verb|'选择的原因'|:字符串类型,默认为空字符串
\end{itemize}

返回:字符串类型

\section{询问选择玩家}

\begin{verbatim}
<表达式> 从 <表达式> 选择一名角色
\end{verbatim}

玩家类型;数组(玩家类型) \\

额外参数:

\begin{itemize}
  \item \verb|'选择的原因'|:字符串类型,默认为空字符串
  \item \verb|'提示框文本'|:字符串类型,默认为默认的提示文本
  \item \verb|'可以点取消'|:布尔类型,默认为\verb|真|
  \item \verb|'提示技能发动'|:布尔类型,默认为\verb|假|
\end{itemize}

返回:玩家类型

\section{询问发动技能}

\begin{verbatim}
<表达式> 选择发动 <字符串>
\end{verbatim}

玩家类型

<字符串>是技能的中文名字,且只能是本文件中已经定义的技能。定义的先后顺序不重要

返回:布尔类型

\section{获得卡牌}

\begin{verbatim}
<表达式> 获得卡牌 <表达式>
\end{verbatim}

玩家类型;卡牌类型 \\

额外参数:

\begin{itemize}
  \item \verb|'公开'|:布尔类型,默认为\verb|真|
\end{itemize}

\section{拥有技能}

\begin{verbatim}
<表达式> 拥有技能 <字符串>
\end{verbatim}

玩家类型

<字符串>是技能的中文名字,且只能是本文件中已经定义的技能。定义的先后顺序不重要

返回:布尔类型

\section{因发动技能而弃牌}

\begin{verbatim}
<表达式> 因技能 <字符串> 弃置卡牌 <表达式>
\end{verbatim}

玩家类型;卡牌数组

<字符串>是技能的中文名字,且只能是本文件中已经定义的技能。定义的先后顺序不重要

本语句只能用在主动技的效果中。

\section{主动技的发动次数}

\begin{verbatim}
<表达式> 发动主动技 <字符串> 的次数
\end{verbatim}

玩家类型;卡牌数组

<字符串>是技能的中文名字,且只能是本文件中已经定义的技能。定义的先后顺序不重要

这种办法只能获取当前阶段里面发动那个技能的次数。如果想要做一回合发动多少次的技能,请使用隐藏标记实现。

\section{令角色弃牌}

\begin{verbatim}
<表达式> 弃置 <表达式> 张牌
\end{verbatim}

玩家类型;数字类型 \\

额外参数:

\begin{itemize}
  \item \verb|'技能名'|: 发起这次弃牌的技能名,默认为空字符串。
  \item \verb|'最小弃置数量'|: 数字类型,默认为要求弃牌数量的值。
  \item \verb|'可以点取消'|: 布尔类型,是否可以点击取消拒绝弃牌,默认为\verb|假|。
  \item \verb|'可以弃装备'|: 布尔类型,是否可以弃置装备牌,默认为\verb|真|。
  \item \verb|'提示信息'|: 字符串类型,默认为空字符串(默认的提示信息)。
  \item \verb|'卡牌正则'|: 字符串类型,默认为无限制。
\end{itemize}

返回类型:卡牌数组,即目标角色弃置了的牌,可能是空的数组。

\section{播放台词}

\begin{verbatim}
  <表达式> 说出 <字符串> 的台词
\end{verbatim}

如果有多个编号完的音频还需要选择,在后面加上\{'音频编号': <编号>\}。
