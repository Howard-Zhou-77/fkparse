\chapter{所有的动作语句}

先列出语句格式,再依次说明表达式类型需求,最后说返回值(如果有)。

\section{摸牌}

\begin{verbatim}
<表达式> 摸 <表达式> 张牌
\end{verbatim}

玩家类型;数字类型

\section{失去体力}

\begin{verbatim}
<表达式> 失去 <表达式> 点体力
\end{verbatim}

玩家类型;数字类型

\section{失去体力上限}

\begin{verbatim}
<表达式> 失去 <表达式> 点体力上限
\end{verbatim}

玩家类型;数字类型

\section{造成伤害}

\begin{verbatim}
<表达式> 对 <表达式> 造成 <表达式> 点伤害
\end{verbatim}

玩家类型;玩家类型;数字类型

\section{受到伤害}

\begin{verbatim}
<表达式> 受到 <表达式> 点伤害
\end{verbatim}

玩家类型;数字类型

\section{回复体力}

\begin{verbatim}
<表达式> 回复 <表达式> 点体力
\end{verbatim}

玩家类型;数字类型

\section{回复体力上限}

\begin{verbatim}
<表达式> 回复 <表达式> 点体力上限
\end{verbatim}

玩家类型;数字类型

\section{获得技能}

\begin{verbatim}
<表达式> 获得技能 <表达式>
\end{verbatim}

玩家类型;字符串类型

\section{失去技能}

\begin{verbatim}
<表达式> 失去技能 <表达式>
\end{verbatim}

玩家类型;字符串类型

\section{获得标记}

\begin{verbatim}
<表达式> 获得 <表达式> 枚 <字符串> [隐藏] 标记
\end{verbatim}

玩家类型;数字类型

<字符串>表示标记的名称。

注:“隐藏”可填可不填,如果写了是隐藏标记的话,获得标记将不会显示在战报中,也没有图片,可以借此实现技能发动次数控制等。

\section{失去标记}

\begin{verbatim}
<表达式> 失去 <表达式> 枚 <字符串> [隐藏] 标记
\end{verbatim}

玩家类型;数字类型

\section{统计标记数量}

\begin{verbatim}
<表达式> <字符串> [隐藏] 标记数量
\end{verbatim}

玩家类型

返回:数字类型

\section{询问选择选项}

\begin{verbatim}
<表达式> 从 <表达式> 选择一项
\end{verbatim}

玩家类型;数组(字符串类型)

返回:字符串类型

\section{询问选择玩家}

\begin{verbatim}
<表达式> 从 <表达式> 选择一名角色
\end{verbatim}

玩家类型;数组(玩家类型)

返回:玩家类型

\section{询问发动技能}

\begin{verbatim}
<表达式> 选择发动 <字符串>
\end{verbatim}

玩家类型

<字符串>是技能的中文名字,且只能是本文件中已经定义的技能。定义的先后顺序不重要

返回:布尔类型

\section{获得卡牌}

\begin{verbatim}
<表达式> 获得卡牌 <表达式>
\end{verbatim}

玩家类型;卡牌类型

\section{拥有技能}

\begin{verbatim}
<表达式> 拥有技能 <字符串>
\end{verbatim}

玩家类型

<字符串>是技能的中文名字,且只能是本文件中已经定义的技能。定义的先后顺序不重要

返回:布尔类型

