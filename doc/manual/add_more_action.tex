\chapter{添加更多的动作语句}

动作语句曾经每一种作为一个单独的类出现,但自从函数功能被实现后,向fkparse中加入自定义action已经不是难事。下面带着例子简要介绍一下向fkparse添加动作语句的办法。

\section{在fkparser.lua中添加接应的函数}

下面以添加broadcastSkillInvoke为例。要调用这个函数需要room对象、技能名、音频的编号。由于fkparse中没有Room类型(因为没必要让用户知道这种类型的存在),因此改为需要ServerPlayer类型。

据此在fkparser.lua下面的\verb|fkp.functions|表中添加以下内容:

\begin{verbatim}
broadcastSkillInvoke = function(player, skill, index)
  player:getRoom():broadcastSkillInvoke(skill, index)
end,
\end{verbatim}

\section{在builtin.c中添加内置函数}

然后进入到builtin.c的\verb|builtin_func|数组下,在两个NULL那一行上面插入以下内容:

\begin{verbatim}
{"__broadcastSkillInvoke", "fkp.functions.broadcastSkillInvoke", TNone, 3, {
  {"玩家", TPlayer, false, {.s = NULL}},
  {"技能名", TString, false, {.s = NULL}},
  {"音频编号", TNumber, true, {.n = -1}},
}},
\end{verbatim}

这些东西是基于前面对结构体的定义而编写的。第一个字符串表示fkparse内部使用的函数名称,第二个表示将要被翻译成的名称,第三个表示函数的返回值类型,第四个表示函数的参数数量,接下来的数组表示函数的各种参数(最多10个参数)。

参数的数组中,参数的顺序必须合乎在fkparser.lua中所定义的那样,各个参数的类型也一样,至于参数的名称随意,但是像“skill”这种需要被翻译的技能名的话,参数的名称中必须包含“技能”这两个字,不然的话程序会将用户输入参数的字符串原封不动复制到lua中。

关于每个参数,第一个字符串是参数的名称,第二个是ExpVType枚举类型,表示参数的类型,第三个布尔类型表示参数是不是有默认值。第四个联合体中,如果没有默认值,那么一律\verb|{.s = NULL}|,否则根据他的类型决定初始化字符串s或者整数n。这里“音频编号”的默认值是-1,所以就那么填写了。

\section{设计语法规则}

下面来设计新action的语法规则。

首先,语法句子中必须包含好所有非默认参数,然后语法尽可能要简明易懂,但一定不能让分析器出现移入/归约冲突之类的错误。(只要不去使用“<表达式> 的”这样的组成,这种冲突通常可以避免)。

比如将broadcastSkillInvoke的语法设计为:

\begin{verbatim}
<表达式> 说出 <字符串> 的台词
\end{verbatim}

这种语法看似可行,包含了参数ServerPlayer和String。

\section{补充词法单元}

绝大多数情况下设计的语法包含有fkparse无力处理的词语,比如这里的词语“说出”和“台词”,在lex.l没有定义过。

总之去lex.l中间的部分找一块地,然后输入定义词法单元的内容:

\begin{verbatim}
"说出"    { return SPEAK; }
"台词"    { return ACT_LINE; }
\end{verbatim}

return后随便跟一个全大写的就行了,只要不和lex.l中已经有的重复。然后前面字符串也不能是lex.l已有的,也就是说不能重复定义词法规则。

然后现在return后面跟随着的全大写其实尚未定义,那么去grammar.y下面,将未定义词语加入:

\begin{verbatim}
%token INVOKE HAVE
%token BECAUSE THROW TIMES
// 注释:前面两行是已有的,仅用来提示位置,下面一行是新加的
%token SPEAK ACT_LINE
\end{verbatim}

这样就完成了词法单元的补充。

\section{添加设计的文法}

接下来就是在grammar.y中将设计好的文法加入。首先决定好文法对应非终结符号的名字,自然就叫broadcastSkillInvoke了。

去grammar.y中“\%\%”前面一行,输入文法的定义:

\begin{verbatim}
broadcastSkillInvoke  : exp SPEAK STRING FIELD ACT_LINE
                      ;

// 注释:下面的%%和函数定义仅用来指示位置
%%

static int yyreport_syntax_error(const yypcontext_t *ctx) {
// ...
\end{verbatim}

接下来是为文法加入动作语句,告诉程序该生成怎么样的\verb|func_call|。参考前面已经写好的,在合适位置写下如下内容:

\begin{verbatim}
broadcastSkillInvoke  : exp SPEAK STRING FIELD ACT_LINE {
                          tempExp = newExpression(ExpStr, 0, 0, NULL, NULL);
                          tempExp->strvalue = $3;
                          $$ = newFunccall(
                                strdup("__broadcastSkillInvoke"),
                                newParams(2, "玩家", $1, "技能名", tempExp)
                              );
                        }
                      ;
\end{verbatim}

这里稍微说明一下动作语句:动作语句其实就是C语句,但是加入了\$\$和\$n这样的符号。\$\$表示的是当前文法左半边,\$n表示的是文法右半边的第n个符号。比如这段代码中的\$3就表示第三个符号,即STRING,\$1就是第一个符号exp了。

newFunccall的意思是创建一个新的函数调用,这是fkparse用来分析的内部结构体之一。第一个参数字符串必须用strdup复制一次(内存管理方便),第二个参数接受一个哈希表,表示这个函数调用的参数。我已经写好了一个方便的函数newParams,直接构造需要的哈希表,第一个参数是调用时的参数数量,往后就是每一个参数,名字、值、名字、值...其中值必须是ExpressionObj *类型,所以这边需要手动造个。\\

至此还有最后一步:加入新文法的类型声明和推导规则。这里创建的是新action,自然要从\verb|action_stat|推导出来。

添加类型声明:

\begin{verbatim}
  %type <func_call> throwCardsBySkill getUsedTimes
+ %type <func_call> broadcastSkillInvoke

  %type <exp> exp prefixexp opexp
\end{verbatim}

添加推导规则:

\begin{verbatim}
            | throwCardsBySkill { $$ = $1; yycopyloc($$, &@$); }
            | getUsedTimes { $$ = $1; yycopyloc($$, &@$); }
+           | broadcastSkillInvoke { $$ = $1; yycopyloc($$, &@$); }
            ;
\end{verbatim}

前面带加号的行表示这是插入的新行。

\section{编写测试例并测试}

去basic.txt的某处将语句写进去:

\begin{verbatim}
  使用后: 你摸1张牌。
+   你说出"生有"的台词。
+   你说出"生有"的台词{'音频编号':1}。
\end{verbatim}

重新编译出可执行文件(参考README.md),然后编译一下新的basic.txt,打开生成的basic.lua看看效果:

\begin{verbatim}
on_use = function(self, player, targets, cards)
  local room = player:getRoom()
  local locals = {}
  global_self = self

  fkp.functions.drawCards(player, 1)
  fkp.functions.broadcastSkillInvoke(player, 'basic_s_6', -1)
  fkp.functions.broadcastSkillInvoke(player, 'basic_s_6', 1)
end,
\end{verbatim}

至此我们已经成功的新建了一个action语句,剩下的就是实机测试了,别忘了把改过了的fkparser.lua也复制进游戏里面。

\section{补充文档}

新的动作语句不能没有文档,切记最后去\verb|all_action.tex|中把新建的语法补充进去。\\

附注:本章中介绍的内容已经在代码中实际体现,请随意参考。

